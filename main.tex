\documentclass[a4paper]{article}
\usepackage{fancyhdr}
\usepackage[includeheadfoot,left=1in, right=0.5in, top=0.5in, bottom=0.5in]{geometry}
\usepackage{lastpage}
\usepackage{extramarks}
\usepackage[usenames,dvipsnames]{color}
\usepackage{graphicx}
\usepackage{listings}
\usepackage{courier}
\usepackage{tikz}
\usepackage{color}
\usepackage{float}
\usepackage{url}
\usepackage{subfigure}
\usepackage{varwidth}
\usepackage{caption}
\usepackage{multirow}
\usepackage[pdfborder={0 0 0}]{hyperref}
\usepackage[compact,small]{titlesec}
\usepackage{microtype}
\usepackage{verbatim}
\usepackage{booktabs}
\usepackage{indentfirst}
\usepackage{enumitem}
\usepackage{pdfpages}

\captionsetup[sub]{labelsep=newline}

% line spacing
\linespread{2.0}

% bold item
\let\origitem\item
\renewcommand{\item}{\normalfont\origitem}
\newcommand{\bolditem}{\small\bfseries\origitem}

% tilde
%\newcommand{\small_tilde}{\raise.17ex\hbox{$\scriptstyle\sim$}}

% indent item
\newcommand{\indentitem}{\setlength\itemindent{24pt}}

% perfect tilde
\newcommand{\tildep}{\raise.17ex\hbox{$\scriptstyle\sim$}}

\parskip = 0.5\baselineskip
\setlength{\belowcaptionskip}{-\baselineskip}

\captionsetup{font=scriptsize}
\captionsetup{labelfont=bf}

\pagestyle{fancy}
\rhead{\fancyplain{}{\rightmark }}
\lhead{\fancyplain{}{\leftmark }}
\rfoot{Page\ \thepage\ of \protect\pageref{LastPage}}
\cfoot{}
\renewcommand\headrulewidth{0.4pt}
\renewcommand\footrulewidth{0.4pt}

% make verbatim text small
\makeatletter
\g@addto@macro\@verbatim\small
\makeatother

%\setlength\parindent{0pt} % Removes all indentation from paragraphs
%\setlength\parindent{24pt}

\definecolor{sh_comment}{rgb}{0.12, 0.38, 0.18 } %adjusted, in Eclipse: {0.25, 0.42, 0.30 } = #3F6A4D
\definecolor{sh_keyword}{rgb}{0.37, 0.08, 0.25}  % #5F1441
\definecolor{sh_string}{rgb}{0.06, 0.10, 0.98} % #101AF9

%\sectionfont{\centering}
\lstset{
    language=vhdl,
    xleftmargin=.25in,
    xrightmargin=.25in,
    numbers=left,
    numberstyle=\tiny,
    frame=tb,
    showstringspaces=false,
    captionpos=b,
    stringstyle=\color{sh_string},
    keywordstyle = \color{sh_keyword}\bfseries,
    commentstyle=\color{sh_comment}\itshape,
    basicstyle=\small\sffamily,
    %numbersep=-5pt,
    belowskip=\baselineskip,
    aboveskip=\baselineskip
}
\usepackage{authblk}

\title{
    \vspace{2in}
    \textbf{Cluster Headaches and Other Stories \\}
    \vspace{2in}
}

\author{Eric Carver}

\affil{carverer@mail.uc.edu}
    \vspace{10pt}

\affil{(513) 207-6501}
    \vspace{2.0in}

\titleformat*{\section}{\large\normalfont}

\begin{document}

%\includepdf{}
\maketitle
\thispagestyle{empty}
\newpage
\thispagestyle{empty}
\parskip = 0.2\baselineskip
\newpage
\thispagestyle{empty}
\section*{\textbf{Abstract}}
\newpage
\thispagestyle{empty}
\section*{\textbf{Acknowledgements}}
\newpage
\thispagestyle{empty}
\tableofcontents
\newpage
\thispagestyle{empty}
\listoffigures
\listoftables
\parskip = 0.5\baselineskip
\newpage

\section{\textbf{Introduction}}


\subsection{\textbf{Research Statement}}

\subsection{\textbf{Thesis Overview}}

%    \begin{figure}[H]
%        \centering
%        \includegraphics[width=\textwidth,height=\textheight,keepaspectratio]{../../magic/pics/magic_layout_32_bit.png}
%        \caption{\textbf{32 Bit Layout - Magic}}
%        \label{fig:gg}
%    \end{figure}

%\section{\textbf{Work Division}}
%    \begin{table}[H]
%        \centering
%        \begin{tabular}{l | p{8cm}}
%            \hline
%            \textbf{Student}   & \textbf{Task} \\ \hline
%            \midrule
%                Both        & Modified Pin-out Diagram \\
%                Both        & Magic Layout \\
%                Both        & IRSIM \\
%                Both        & VHDL \\
%                Both        & Modified Floor Plan
%        \end{tabular}
%        \caption{\textbf{Task Assignment}}
%    \end{table}

\newpage
\section{\textbf{Background}}

\subsection{\textbf{Infiniband and RoCE}}

\href{www.infinibandta.org}{InfiniBand}\texttrademark (IB) was introduced in
1999 to meet the new demand of data-intensive applications in high-end computing
environments \cite{InfiniBandTABase-07}. Its characteristic high bandwidth and
low latency make it a popular interconnect technology for computing solutions
intended for fine-grained parallel simulation. For high-end Beowulf Clusters,
message latency as low as one microsecond is provided through the two-pronged
approach of using InfiniBand networking hardware and the lightweight IB
transport layer. However, the high cost and lack of inter-fabric compatibility
of such hardware have driven the development of solutions based on the
ubiquitous Ethernet link layer \cite{roce-announce}.

Although IB is a full first-order interconnect solution, its transport layer is
particularly attractive to researchers who desire affordable high performance
computing. It specifies four transport types: Reliable Connection (RC), Reliable
Datagram (RD), Unreliable Connection (UC), and Unreliable Datagram (UD)
\cite{InfiniBandTABase-07}. All four types support channel messaging semantics,
wherein messages are passed using send and receive calls
\cite{InfiniBandTABase-07}. The RC and UC types support memory semantics, also
known as Remote Direct Memory Access (RDMA). This approach allows an initiator
computing node to access the memory of a remote process directly, without any
significant effort on the part of the remote node \cite{sur-11}.

RDMA over Converged Ethernet (RoCE), also known as Infiniband over Ethernet
(IBoE), is an attempt to provide the reliable, low-latency InfiniBand transport
services over a converged Ethernet fabric already present in most data centers
\cite{InfiniBandTARoCE-10}, \cite{roce-announce}. This enables properly
configured networks to carry IB traffic without investing in entirely new
physical hardware. Message latency can be under 2 microseconds when used in a
lossless 40 Gigabit Ethernet network \cite{vienne-12}. RoCE has found a niche in
commercial data centers, especially those that support financial operations such
as high frequency trading. Unfortunately, specialized network adaptors are still
required to take advantage of its features. Additionally, these solutions
generally require switches that support Data Center Bridging (DCB)
\cite{InfiniBandTARoCE-10}. This creates an opportunity for a solution that can
work with existing lossy Ethernet networks.

%% Could talk about iWARP here, but probably don't need to

System Fabric Works, Inc., has created a pure software implementation of RoCE
called \href{http://www.systemfabricworks.com/downloads/roce}{rxe}. rxe is a
Linux driver that implements the full IB transport over any Ethernet adaptor
\cite{pearson-10}. Message latency can be as low as 10
microseconds\cite{pearson-10}. The
\href{http://support.systemfabricworks.com/downloads/rxe/}{most recent relase of
the driver} is in the from of patches for the mainline Linux kernel. We chose
to port this implementation to the ODROID platform for our studies in low-cost
Beowulf Clusters.

\subsection{\textbf{Driver Model}}

%% Talk about polling here
%% TODO: Get info from Doug (EHCI, XHCI, polling implementation)

\subsection{\textbf{ODROID Platform}}

\subsubsection{\textbf{ODROID-U2}}

\subsubsection{\textbf{ODROID-XU}}

\subsection{\textbf{Related Work}}

%% ERIC: do we know of any related work?  even publications talking about
%% IBoE/RoCE performance on the x86 platform would be worth reviewing.

Although performance profiling of hardware IB and RoCE is extensive
\cite{subamaroni-09}, \cite{vienne-12}, studies on software RoCE are
rare. Robert J. Lancaster \cite{lancaster-10} experimented with the rxe
driver. He found that rxe message latency ranged from 14\% to 81\% lower than
TCP/IP using a Realtek 8168B Gigabit Ethernet adaptor with the r8169 driver.

%% Can't really find anything else on softroce



\bibliographystyle{abbrv}
\bibliography{refs}

\end{document}
