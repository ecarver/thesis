\documentclass[a4paper]{article}
\usepackage{fancyhdr}
\usepackage[includeheadfoot,left=1in, right=0.5in, top=0.5in, bottom=0.5in]{geometry}
\usepackage{lastpage}
\usepackage{extramarks}
\usepackage[usenames,dvipsnames]{color}
\usepackage{graphicx}
\usepackage{listings}
\usepackage{courier}
\usepackage{tikz}
\usepackage{color}
\usepackage{float}
\usepackage{url}
\usepackage{subfigure}
\usepackage{varwidth}
\usepackage{caption}
\usepackage{multirow}
\usepackage[pdfborder={0 0 0}]{hyperref}
\usepackage[compact,small]{titlesec}
\usepackage{microtype}
\usepackage{verbatim}
\usepackage{booktabs}
\usepackage{indentfirst}
\usepackage{enumitem}
\usepackage{pdfpages}

\captionsetup[sub]{labelsep=newline}

% line spacing
\linespread{2.0}

% bold item
\let\origitem\item
\renewcommand{\item}{\normalfont\origitem}
\newcommand{\bolditem}{\small\bfseries\origitem}

% tilde
%\newcommand{\small_tilde}{\raise.17ex\hbox{$\scriptstyle\sim$}}

% indent item
\newcommand{\indentitem}{\setlength\itemindent{24pt}}

% perfect tilde
\newcommand{\tildep}{\raise.17ex\hbox{$\scriptstyle\sim$}}

\parskip = 0.5\baselineskip
\setlength{\belowcaptionskip}{-\baselineskip}

\captionsetup{font=scriptsize}
\captionsetup{labelfont=bf}

\pagestyle{fancy}
\rhead{\fancyplain{}{\rightmark }}
\lhead{\fancyplain{}{\leftmark }}
\rfoot{Page\ \thepage\ of \protect\pageref{LastPage}}
\cfoot{}
\renewcommand\headrulewidth{0.4pt}
\renewcommand\footrulewidth{0.4pt}

% make verbatim text small
\makeatletter
\g@addto@macro\@verbatim\small
\makeatother

\setlength\parindent{0pt} % Removes all indentation from paragraphs
%\setlength\parindent{24pt}

\definecolor{sh_comment}{rgb}{0.12, 0.38, 0.18 } %adjusted, in Eclipse: {0.25, 0.42, 0.30 } = #3F6A4D
\definecolor{sh_keyword}{rgb}{0.37, 0.08, 0.25}  % #5F1441
\definecolor{sh_string}{rgb}{0.06, 0.10, 0.98} % #101AF9

%\sectionfont{\centering}
\lstset{
    language=vhdl,
    xleftmargin=.25in,
    xrightmargin=.25in,
    numbers=left,
    numberstyle=\tiny,
    frame=tb,
    showstringspaces=false,
    captionpos=b,
    stringstyle=\color{sh_string},
    keywordstyle = \color{sh_keyword}\bfseries,
    commentstyle=\color{sh_comment}\itshape,
    basicstyle=\small\sffamily,
    %numbersep=-5pt,
    belowskip=\baselineskip,
    aboveskip=\baselineskip
}
\usepackage{authblk}

\title{
    \vspace{2in}
    \textbf{Experiments with Hardware-based Transactional Memory in Parallel
    Simulation \\}
    \vspace{2in}
}

\author{Joshua Hay}

\affil{hayja@mail.uc.edu}
    \vspace{10pt}
    
\affil{(513) 607-4929}
    \vspace{2.0in}

\titleformat*{\section}{\large\normalfont}

\begin{document}

%\includepdf{}
\maketitle
\thispagestyle{empty}
\newpage
\thispagestyle{empty}
\parskip = 0.2\baselineskip
\newpage
\thispagestyle{empty}
\section*{\textbf{Abstract}}
\newpage
\thispagestyle{empty}
\section*{\textbf{Acknowledgements}}
\newpage
\thispagestyle{empty}
\tableofcontents
\newpage
\thispagestyle{empty}
\listoffigures
\listoftables
\parskip = 0.5\baselineskip
\newpage

\section{\textbf{Introduction}}

The advent of multi-core programming introduced a new avenue for increased
performance and scalability through multi-threaded programming.  However, this
avenue came with a toll: synchronization mechanisms for access to critical
sections.  The critical section is segment of code accessing a shared resource
that can only be executed by one thread at any given time \cite{os_concepts}.
This effectively results in the serialization of thread execution on this
segment of code. \par

For example, a multi-threaded application is designed to operate on a single,
shared memory data structure.  For the sake of simplicity, programmers typically
use coarse-grained locking techniques to protect the critical section, e.g. a
single atomic lock for the entire data structure.  A thread will lock all other
threads out of the entire data structure until it has completed its task, thus
forcing the threads to execute this segment of code serially. \par

Coarse-grained locking is a pessimistic synchronization mechanism; accesses to
shared resources are statically serialized by the programmer \cite{}.  It
prevents multi-threaded applications from executing at their full potential as
provided by the underlying hardware.  More importantly, it increases  
lock contention, resulting in performance penalties rather than performance
gains. \par



\subsection{\textbf{Research Statement}}

\subsection{\textbf{Thesis Overview}}

%    \begin{figure}[H]
%        \centering
%        \includegraphics[width=\textwidth,height=\textheight,keepaspectratio]{../../magic/pics/magic_layout_32_bit.png}
%        \caption{\textbf{32 Bit Layout - Magic}}
%        \label{fig:gg}
%    \end{figure}

%\section{\textbf{Work Division}}
%    \begin{table}[H]
%        \centering
%        \begin{tabular}{l | p{8cm}}
%            \hline
%            \textbf{Student}   & \textbf{Task} \\ \hline
%            \midrule
%                Both        & Modified Pin-out Diagram \\
%                Both        & Magic Layout \\
%                Both        & IRSIM \\
%                Both        & VHDL \\
%                Both        & Modified Floor Plan
%        \end{tabular}
%        \caption{\textbf{Task Assignment}}
%    \end{table}

\newpage
\bibliographystyle{abbrv}
\bibliography{refs}

\end{document}
